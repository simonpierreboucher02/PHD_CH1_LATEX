\section{Introduction} \label{sec:introduction}
%MH    MH     MH     il faut introduire des éléments du randr dans l'introduction sinon rien ne change et l'arbitre va penser qu'on a rien fait. 
Despite investment outflows in 2009 and 2014-2016,\footnote{These outflows are explained by the Great Recession and a global commodity slump, respectively.} commodities remain a popular asset class among non-traditional market participants such as hedge funds and index traders.  Indeed, 2022 may be a record year for commodity trading profit.\footnote{See, for example, the Wall Street Journal, September 9th 2022, ``Wall Street's Commodity Traders on Track to Break Profit Records''.} The financialization of commodities refers to important market and regulatory changes that have affected how commodities (futures, options, swaps and physicals) are traded by institutional and other non-traditional investors.  Financialization is controversial: Testimony given by \citet{masters2009testimony} to the Commodities Futures Trading Commission (CFTC) argued  that institutional investors disrupted commodity markets in the mid-2000s through the use of strategies intended for financial securities.\footnote{As we explain later, however, this claim has been challenged by empirical research \citep*{irwin2011index,irwin2012testing,irwin2012financialization}.} This paper investigates financialization in commodity markets through the lens of macroeconomic announcement releases.
 %MH    MH    MH    ici il manque une discussion sur comment on mesure la financiarisation.

Commodity financialization has coincided with a commodity bull cycle \citep*{humphreys2010great}, during which traditional market participants (i.e., those who produce or process the commodity) expressed fears that prices would be distorted from their fundamental values and that volatility increases would make hedging more costly. 
Research has found that increases in speculation and long-only index positions in commodity markets are linked to increases in volatility and correlations between commodities.\footnote{Helpful surveys of this large literature include \citet{boyd2018update} and \citet{cheng2014financialization}.}
 Theory suggests that financialization could affect commodity futures markets through risk sharing and information discovery \citep*{cheng2014financialization}. Indeed, investors can either provide liquidity to meet the hedging needs of other traders or consume liquidity when they trade for their own needs \citep*{kang2020tale}, thereby influencing liquidity risk. In addition, price discovery in commodity markets is affected by informational frictions concerning supply, demand and inventories. Price discovery tends to occur on futures markets, as spot or cash markets are more decentralized \citep*{garbade1983price}. Since new players in commodities mostly take positions in derivatives rather than physicals, financialization could alter how these markets incorporate new information.   This paper shows that we can assess the impact of financialization by measuring how information discovery is affected by the arrival of new players in commodity markets. 


To investigate how financialization affects commodities, we analyze how prices and variance change in response to unexpected macroeconomic news. We use high frequency data ranging from April 2nd, 2007 to October 31st, 2023, and a methodology similar to \citet*{andersen2007real} and \citet*{kurov2019price}. We focus on six major commodity futures. Using intraday data allows us to more accurately measure the effect of news on volatility and returns. Moreover, we avoid a common criticism of event study methods, namely that daily frequency data may attribute the effect of an announcement to another, concurrent market event \citep*[see e.g., ][]{kothari2007econometrics}. To achieve this objective, we construct financialization indices based on the definition given in the literature. These measures allow us to assess financialization, taking into account the role of speculation and the intensity of trade. We also provide a sub-periodic analysis documenting the stability of our result during periods of financial crisis and zero lower bound. We also shed light on the structure of the commodities market by breaking down the impact of several market players: commercial and non-commercial (swap dealers and fund managers).

Our first contribution is to present new insights on financialization by bridging this line of research with the macro announcements literature. The announcements literature has shifted from using daily data to high-frequency data and has examined a variety of markets such as bonds \citep{andersen2007real, hu2013noise, balduzzi2001economic,lee1995oil, hautsch2011impact, kurov2019price}, stocks \citep{andersen2007real,bernile2016can,kurov2019price} and foreign exchange rates \citep{lee1995oil,andersen2003micro}. However, the use of high-frequency data is less common in commodity futures research \citep{couleau2020corn}. We argue that the literature's use of a lower sampling frequency for futures price returns could help explain why there is no  consensus as to the impact on volatility \citep*{tang2012index,brunetti2016speculators,irwin2012testing,stoll2010commodity,alquist2013role}. Indeed, the use of daily or weekly data results in a smaller sample size and reduces the power of statistical tests \citep*{irwin2009devil}.  

In so doing, our paper also builds on the literature on information transmission in markets. \citet*{goldstein2022commodity} develop a theory of how financial participants in commodity markets affect futures price informativeness, bias, and comovement. Their model predicts that financialization, defined here as increasing the number of  speculators relative to  hedgers in a  market, initially improves price efficiency but eventually makes it worse. Our results further contribute to research on the impact of institutional investors. For instance, \citet*{brunetti2016speculators} find that hedge funds add liquidity to commodity markets, resulting in more efficient prices, while merchant positions are linked to greater volatility in crude oil and natural gas markets.


Our second contribution is to provide disaggregated results, where fund managers and swap dealers are analyzed separately, and to consider more informative measures of financialization than what is typically used.   Prior studies often split the sample into pre- and post-financialization periods, using 2004\footnote{Although the Commodity Futures Trading Commission passed the Commodity Futures Modernization Act in 2000, the literature generally agrees that 2004 marks the beginning of financialization.}   as  the break point.\footnote{Some examples include \citet{buyukcsahin2010matters, kilian2014role,brunetti2016speculators,irwin2012financialization,stoll2010commodity,alquist2013role}.} Instead, we use several measures to let financialization be time-varying and commodity-specific.   Overall, our research sheds new light on the resolution of uncertainty, market efficiency and information transmission, and the anticipation of macroeconomic announcements. 
%when it comes to financialization
%the intensity of  ... measures
%This approach could, however, neglect time-varying levels of financial trader activity. 
%These measures are  available individually for each commodity and therefore have the advantage of not treating commodity markets as a monolith.
%Indeed, financialization  varies not only over time but also between commodities.
%\citep[e.g., ][]



We summarize our findings as follows: First, financialization contributes to information diffusion and price discovery. The impact of a surprise following a macroeconomic announcement is generally dampened when a commodity is more financialized. Thus, it appears that commodity markets are better informed since macro news generate less of a  shock. This outcome should  benefit traditional commodity market participants. The dampening effect we find is stronger for pro-cyclical commodities such as crude oil or natural gas than for gold, which is a safe haven asset.%. In contrast, the evidence is weaker in the case of gold, a safe haven asset.% that is less affected by macroeconomic announcements. 
%who typically hold short position in the underlying asset.

Second, we find that money managers reduce volatility and contribute more to price discovery when there is a macro announcement. This result is consistent with the idea that money managers are more informed investors, given their role in the market \citep{fishe2012identifying}. In contrast, swap dealers also contribute to price discovery but are linked to  increases in volatility after macro announcements. If we aggregate all trader categories, financialization seems to reduce volatility in commodity markets. Robustness checks show that our main findings are unchanged if we consider alternative volatility estimators, different econometric specifications, or if we change the financialization proxy.  Thus, our results suggest that financialization is helpful to commodity markets. 


The paper's implications on financialization go beyond the efficiency of commodity markets. With the ongoing shift towards green energy, how might financialization affect the growth of sustainable energy markets? If financialization means less volatile commodity and energy markets, we could see increased investment in the green energy transition.\footnote{This follows from real options theory, as higher volatility discourages investment by increasing the value of waiting to invest \citep*{kellogg2014effect}.}  Beyond the energy sector, demand has grown for the metals and minerals needed to build a renewable energy infrastructure \citep*{knuth2018breakthroughs}. By broadening the investor base, financialization could play a role in creating more sustainable, affordable and accessible raw energy markets.
%Therefore, greater stability in commodity markets could encourage more investment in the green energy transition.
%At present, the literature focuses on the impact of financialization on energy firms
%changes could occur in how financialization affects futures contracts or in the relationship between commodity futures and spot prices
%As the economy gradually moves away from fossil fuels, the relationship between financial actors and commodity markets could change. 