\section{Results} \label{sec:result}
%We now present the results of regressions explaining returns
This section presents and discusses our baseline empirical results, as well as some robustness checks. Before describing the results, we review what is expected from macroeconomic announcements and their economic implications.
%Before we look at the results we have, it is important to consider what expectations we have about the interpretation of a macroeconomic announcement and, more importantly, what it implies for the economy overall.
   \citet*{boehm2020us} track the behavior of different financial assets following releases for different U.S. announcements. Their results help us tell for which  macro announcements a positive sign on the surprise variable would signal a weak economy. In our sample, only Consumer Price Index \citep*{bryan1993consumer,clark1997us} and Initial Jobless Claims \citep*{fleming1997moves,fleming1999price,getz1990barometer} announcement releases would signal a weaker economy when the surprise is positive.  For the other announcements, a positive surprise will be interpreted by investors as signaling a strong economy. %In other words, a better than expected macroeconomic announcement value will be accompanied by a positive surprise variable.

   %when the sign of the surprise variable is positive
%Thus, when the value of the macro announcement is better than expected, the surprise variable will have a negative sign.
 
% \citet*{boehm2020us} track the behavior of different financial assets following releases for different types of announcements in the U.S. market. Their results are useful to help explain which particular macro announcements send a signal of a weak economy when the sign of the surprise variable is positive. In our sample, only announcements of the Consumer Price Index  \citep*{bryan1993consumer,clark1997us} and Initial Jobless Claims \citep*{fleming1997moves,fleming1999price,getz1990barometer} signal a weaker economy when the surprise is positive. For these two, a better than expected value for the release implies that the surprise variable will have a negative sign. For all other announcements in our sample, a positive surprise should be interpreted by investors as signaling a strong economy.
 
 


\subsection{Effect of surprises and financialization on returns}
Table~\ref{tab:return-fin-full} presents the results of regressions to explain high-frequency commodity futures returns in a window following a macro announcement. For brevity, this table presents only results obtained using the NLS financialization variable. Results using either MSCT or Working's $T$ are presented in the online appendix. Our main findings are robust to the choice of proxy. 

To see how macro announcement surprises  affect returns, we consider the $\gamma_m$ coefficient.  We find that for Initial Jobless Claims, where a positive surprise is ``bad news'', the coefficient is negative for crude oil and positive for gold, as expected. For surprises related to CB Consumer Confidence, Advance Retail Sales, ADP Employment, and Pending Home Sales announcements, however, the coefficients are positive for crude oil and negative for gold. These findings support the idea that crude oil is pro-cyclical while gold is considered a safe haven  \citep[see e.g.,][]{lucey2015precious}.
%\textcolor{mypink1}{In addition, in the case of crude oil, we are also adding a surprise in connection with the production announcements made by the Petroleum Exporting Countries Organization (OPEC). Despite the close link between this announcement and oil, the results show that, on average, the announcement made by Opec has no impact on returns and volaitlity.} 
% Il faut dire que c'est un résultat en ligne avec la littérature sinon c'est louche
%For Initial Jobless Claims, the coefficient is negative for crude oil and positive for gold, as expected. Moreover, for surprises linked to the CB Consumer Confidence, Advance Retail Sales, ADP Employment and Pending Home Sales announcements, the coefficients are positive for crude oil and negative for gold. These results are consistent with claims that crude oil is pro-cyclical while gold is seen as a safe haven \citep[see e.g.,][]{lucey2015precious}. 

%%NOUS AVONS RETIRE LA MENTION DES RESULTATS OPEC ANNOUNCEMENTS PUISQUE PAS PRESENTES
%\footnote{In unreported results, for crude oil, we add as a control a surprise variable computed from OPEC production announcements. We find that the OPEC surprise has no significant effect on crude oil futures returns or volatility, and that our main findings are unchanged. As explained above, this is not surprising because OPEC announcements are not reliably time-stamped, so they are not useful in a high-frequency framework.}

As for the other commodities, we find that the effects on copper returns are similar to those of crude oil returns, which is as expected given its role in industrial production. The coefficients have the predicted signs, although they are not significant for all announcements. The results for silver are similar to those for gold. The  surprise coefficient $\gamma_m$ in particular is positive and significant for Initial Jobless Claims. For the other announcements, $\gamma_m$ is negative when it is significant. Lastly,  for natural gas and palladium the coefficients suggest a pro-cyclical response but they are significant for fewer announcements. 

Next, we look at the $\theta$ coefficient, which quantifies the interaction effect between the macro surprise variables and the financialization proxy. For crude oil, gold and silver, $\theta_m$ has the opposite sign to the sign of the macro surprise coefficient $\gamma_m$. 

Thus, an increase in financialization \emph{reduces} the magnitude of the price adjustment due to a macro surprise. Using MSCT, financialization is significant for all commodities when we combine it with surprises for ADP Employment, Durable Goods Orders, and Non-farm Employment announcements. If we use instead the NLS proxy, the effect is significant for surprises in Initial Jobless Claims, ADP Employment, Advance Retail Sales, New Home Sales, and Personal Income. Lastly, using Working's $T$ as a proxy, there is a significant effect for Initial Jobless Claims, ADP Employment, CB Consumer Confidence, Durable Goods Orders, New Home Sales, and Non-farm Employment. 

Thus, we find significant results, especially for employment- and household income-related macro releases. This finding is consistent with \citet*{hordahl2020expectations}, who report that the most important macro announcements are those included in the Employment Report, as they are the most likely to affect asset returns and volatility. 


 %, suggesting that financial actors in crude oil futures markets

%shows that, we see that the gamma coefficient is negative and significant, in the case of crude oil for the crude oil stock announcement. This result is consistent with the fact that an increase in crude oil stocks will certainly have a downward impact on oil prices. The theta coefficient is positive and significant, confirming that financial investors don't seem to attach much importance to commodity-specific announcements. Financial investors seem rather interested in taking the opposite position to that preferred by the markets, confirming their role in adding liquidity to this market.

\subsection{Effect of surprises and financialization on volatility}
	
Table \ref{tab:vol-fin-full} shows the results of equation (\ref{eqn:VarianceEqn}) estimated across commodities.  The macro surprise coefficient $\Theta_m$ is significant for several announcements in the case of crude oil, gold, copper and silver. It is nearly always positive when significant, consistent with the claim that macro surprises usually increase futures volatility. However, the financialization interaction coefficient $\phi_k$ is always negative when it is significant (e.g., for crude oil and copper), suggesting that an increase in financialization dampens the volatility reaction to macro surprises. This result is consistent with \citet{brunetti2016speculators}, who argue that speculation tends to lower volatility rather than increase it.  Moreover, our results are robust to using a non-parametric variance estimator instead of a GARCH model.

We repeat the volatility regressions using instead a financialization variable constructed by Principal Component analysis applied to the three proxies presented earlier. The regression results are shown in table \ref{tab:return-pca-full}. As in the case of return analysis, the coefficients obtained are similar in terms of magnitude and significance.


The results in Tables~\ref{tab:vol-fin-full}, \ref{tab:vol-fin-covid}, and \ref{tab:vol-fin-zlb} highlight how the sensitivity of commodity volatility to economic announcements changes according to macroeconomic conditions. The full sample, running from 2007-2023, serves as a baseline for understanding how commodities generally respond to economic indicators. Crude oil, for instance, shows moderate sensitivity to economic news in this  dataset.

When comparing the full sample to the Covid-19 sub-period, we find that crude oil has a greater sensitivity to specific announcements, such as Non-farm employment. During the Zero Lower Bound (ZLB) era (2007-2015), however, crude oil shows a more muted response to news, supporting a link between monetary conditions and commodity volatility.

Gold consistently reacts to economic announcements across all periods, particularly to Consumer Price Index and Personal Income. Copper does not show reactions as consistently as Gold, but it reacts in particular to  Consumer Price Index and Trade Balance during the ZLB period.
%reinforcing its role as a financial hedge


The effect of financialization, measured by $\phi_m$, is most significant for Gold and Palladium during the Covid period.  Interestingly, $\phi_m$ coefficients for  Copper and Natural Gas are particularly significant during the ZLB period, indicating a greater sensitivity to macroeconomic news during that sub-period.
% PAS SUR This implies a trend toward increased financialization of these commodities in times of economic uncertainty.


Based on the $R^2$ , there is a better model fit during the ZLB period, particularly for Crude Oil and Natural Gas, than during the Covid period. This could indicate a link between monetary policy and the degree to which commodity futures react to macro news.%indicating a more predictable impact of economic announcements during times of constrained monetary policy. The model was moderately effective during the Covid era and provided a broad baseline understanding in the full sample, encompassing the pre-Covid period.

%%IL FAUDRAIT DIRE QUOI EXACTEMENT, 

For all three proxies, the combined effect of macro surprises and financialization leads to economically plausible signs. The results are significant for Advance Retail Sales, Construction Spending, Factory Orders and Non-farm announcements. The only macro announcement surprise with a negative and significant coefficient across commodities is Non-farm Employment, as in \citet{hordahl2020expectations}. These results suggest that commodity financialization increases the efficiency of information discovery that occurs after a surprise in the Employment Report. As for energy inventory announcements, only copper volatility  reacts to crude oil news and none reacts to natural gas news.
%
%Concerning the crude oil and natural gas stock announcements, neither seems to have an impact on conditional volatility.


In summary, the results provide a comprehensive understanding of how commodities respond differently to economic announcements across varied economic conditions.


\subsection{Differences across market participants}


 This section presents additional results for the two individually reported categories of financial participants, namely swap dealers and fund managers. For this analysis, we use the NLS financialization variable, as this proxy allows for separate categories of financial traders. 
The MSCT and Working's $T$ variables cannot be computed in such a way as to separate the different types of financial investors, given the presence of the number of positions of commercial traders in the variable's calculation.


%The upshot of this analysis is to show that our previous finding that a greater participation by financial traders reduces volatility by limiting hedging pressure appears to be attributed to money managers. In contrast, our results suggest that swap dealer activity seems to make hedging pressure worse, which generally leads to higher volatility.

 The regressions results described below include a recession indicator variable defined using the NBER's Business Cycle Dating Committee. This variable is not significant in the return regressions, but it is positive and significant in the volatility regressions. This finding is robust to using the Aruoba-Diebold-Scotti (ADS) Business Conditions Index, published by the Federal Reserve Bank of Philadelphia \citep*{doi:10.1198/jbes.2009.07205}. The ADS variable has the advantage of being continuous rather than dichotomous, and it  is updated more frequently. It is not significant in the estimated returns equation, but it is negative and significant in volatility regressions (as a higher value of ADS indicates a better economic state).


 %\vspace{1cm}
 
Table \ref{tab:return-mm-full} shows results for the return regression using a financialization proxy defined only for money managers. The same results are also presented for the COVID and Zero lower bound sub-periods in tables\ref{tab:return-mm-covid} and \ref{tab:return-mm-zlb} respectively. We find that increased participation by money managers has the same effect as our baseline financialization results. If we take for instance crude oil, the $\gamma_m$ macro surprise coefficient is positive while the  $\theta_m$ financialization coefficient is negative (for significant announcements). Since $\gamma_m$ has the opposite sign of $\theta_m$, we conclude that money managers lower hedging pressure. For swap dealers, table \ref{tab:return-swap-full} shows that the signs for $\gamma_m$ and $\theta_m$ are the same. We obtain the same results using the COVID and zero lower bound subperiods, as shown in tables \ref{tab:return-swap-covid} and \ref{tab:return-swap-zlb}.  Thus, it appears that swap dealers, unlike money managers, make hedging pressure worse and may not help to improve liquidity in commodity futures markets. 




Tables \ref{tab:vol-mm-full} and \ref{tab:vol-swap-full} show results for the conditional variance equation using only money managers and swap dealers, respectively, to capture financialization. Tables \ref{tab:vol-mm-covid} and \ref{tab:vol-swap-covid} show the same results but for the COVID sub-period, while table \ref{tab:vol-mm-zlb} and \ref{tab:vol-swap-zlb} show those for the zero lower bound sub-period. For money managers, we find that the financialization interaction coefficient $\phi_k$ is negative when significant, while for swap dealers it is positive when significant. These results support the economic interpretation of the earlier results for returns. While increased trading by money managers dampens the effect of macro surprises on volatility, a greater presence of swap dealers seems to amplify the effect of surprises on volatility. Thus, hedging pressure is lessened with money managers but worsens with swap dealers. 



%These volatility results further support the finding that money managers appear to lower hedging pressure and volatility, while swap dealers seem to worsen hedging pressure and increase volatility.
%These results tell us that in addition to reducing hedging pressure during a macroeconomic announcement, money managers also seem to contribute to a reduction in volatility.  In contrast, swap dealers seem to worsen hedging pressure during a macroeconomic announcement and increase volatility as well. 
%\vspace{1cm}

While our discussion of the results focuses on crude oil as a benchmark commodity, the evidence for the other pro-cyclical commodities supports our economic arguments. Note that the coefficient signs for gold differ, however, and are instead consistent with a safe haven interpretation \citep*{erb2013golden,bredin2015does}. While gold has characteristics of a commodity and a currency, prior research has focused on how the value of gold increases with investor risk aversion. Indeed, gold can act as a safe haven in periods of economic uncertainty and market turmoil.\footnote{ \citet{baur2010gold} explain what empirical findings would lead to the conclusion than an asset or asset class has safe haven characteristics. For instance, asset returns should be uncorrelated or negatively correlated with other asset returns, and this property should hold only in times of market stress or turmoil.}
%\vspace{1cm}
%Now knowing this properties of gold, f
Therefore, we would expect that in times of crisis financial traders would increase their net long positions in gold futures for reasons unrelated to the actual economics of the gold market. Given the large proportion of gold futures positions held at all times by financial traders, and since non-financial traders can go long or short depending on their hedging needs, there are two possible outcomes. First, when non-financial traders are mostly long in  gold futures, the impact of financial traders is to worsen  hedging pressure  and thus increase volatility. Second, when non-financial traders are mostly short in gold futures, they are more likely to be taking opposite positions to financial traders, which should result in less hedging pressure and lower volatility.
%QUELQUE CHOSE PAS CLAIR DANS CE PARAGRAPHE. RECONCILIER GOING LONG IN CRISIS TIME AVEC GOING LONG POSITIONS INCREASING OVER TIME . ET RECONCILIER BEAUCOUP DE LONG AND FEWER SHORT (CA DOIT ETRE EGAL)

\subsection{Robustness checks}
We present in this section the results of several robustness checks. First, we assess whether our results are maintained in different  macroeconomic environments. The  results of eq. \ref{eqn:MeanEqn} estimated using the Covid sub-sample period are reported in table \ref{tab:return-fin-covid} and for the Zero Lower Bound period (2007 to 2015)  in table \ref{tab:return-fin-zlb}. 

A second robustness check involves estimating the regressions over the full sample (2007 to 2023)  using a financialization proxy constructed by Principal Component Analysis applied to the  NLS, MSCT and Working's T variables. Specifically, the first PCA component is used as a financialization proxy.   Results of the PCA are presented in table \ref{tab:PCA}  while results of the return regressions using the PCA variable are reported  in table \ref{tab:return-pca-full}. The results using the PCA proxy are similar to our baseline results (using NLS), underscoring their robustness.% the coefficients obtained are similar in magnitude and significance, underlining the robustness of the NLS financialization variable.

The empirical analysis of futures returns over different sample periods---the Zero Lower Bound (ZLB) period from 2007 to 2015 (Table~\ref{tab:return-fin-zlb}), and the  Covid-19  period from 2020 to 2022 (Table~\ref{tab:return-fin-covid})---reveals the following insights.

Taking crude oil as a baseline, the immediate return response to a surprise in Initial Jobless Claims, as measured by the $\gamma_m$ coefficient, is negative across all periods. This result shows it is not linked to particular market conditions.  For some announcements such as GDP, however, the \( \gamma_m \) coefficient increases during the Covid period.%However, it's interesting to note that the sensitivity to GDP announcements showed a significant increase during the COVID-19 period, as indicated by a  value of 0.032. This increase may be attributed to heightened market volatility and economic uncertainty during the pandemic.


Gold reacts negatively to ADP Employment surprises in the two sub-sample periods as well as the full sample. The coefficient is more significant during the Covid period, suggesting its safe haven characteristic was more salient during the pandemic.% announcements turned negative during the ZLB and stayed negative through the COVID-19 period, becoming statistically significant. Likewise, the coefficient for "GDP" remained consistently negative but increased in statistical significance during the COVID-19 era, possibly implying that market participants view gold as a safer asset in times of economic uncertainty.
In the case of Silver, the response to a surprise in ADP Employment is negative in the full sample and during the ZLB period, but positive during the Covid period. 
%the coefficients reveal a notable shift. For example, the immediate return response to "ADP Employment" switched from being negative in both the full span and the ZLB period to positive during the COVID-19 period. This flip could signify a change in investor sentiment or trading strategies related to Silver in the context of employment data. 

Overall across commodities, the magnitude of coefficients increases during the Covid sub-period, suggesting a heightened sensitivity to macro surprises. This was a period of considerable uncertainty, but also bottlenecks and supply chain problems in commodity markets. That being said, the main finding is that the results are robust in sub-periods.
%Overall, the coefficients for all commodities across the periods indicate varying degrees of market sensitivity to different economic indicators. During the COVID-19 period, there was generally increased volatility and higher magnitudes for the coefficients, suggesting heightened market sensitivity to announcements. Changes in statistical significance also indicate that the reliability of these announcements in predicting commodity futures returns can be influenced by external conditions, such as global pandemics or monetary policy regimes. This comprehensive analysis highlights the intricacies of the commodity markets and their interaction with economic announcements and financialization, emphasizing the need for a context-sensitive approach for both policymakers and investors.

We also provide robustness results for the conditional variance model (eq. \ref{eqn:VarianceEqn}).  Results for the Covid and ZLB sub-periods are reported in table \ref{tab:vol-fin-covid} and  \ref{tab:vol-fin-zlb} , respectively. Moreover, we re-estimate the model using the PCA financialization proxy.  The results are similar to our baseline findings in terms of coefficient magnitude and significance. 
%We also repeated the analysis on volatility, using a financialization variable constructed by Principal Composent .The regression results are presented in  in the case of return analysis, the coefficients obtained are similar in terms of magnitude and significance.
%we also present the results for the COVID-19 restriction period in 
%%TABLE POUR PCA ANALYSIS VOLATILITY?
%The analysis across Tables~\ref{tab:vol-fin-full}, \ref{tab:vol-fin-covid}, and \ref{tab:vol-fin-zlb} highlights how commodities' sensitivities to economic announcements evolve under varying macroeconomic conditions. The full sample, covering 2007-2023, serves as a baseline for understanding how commodities generally respond to economic indicators. Crude Oil, for instance, shows moderate sensitivity to economic news in this comprehensive dataset.
The results for Crude Oil show that it is more sensitive to macro surprises during the Covid period, and somewhat less sensitive during the ZLB period.
%Upon comparing this with the Covid-19 period, it becomes apparent that Crude Oil's sensitivity to specific announcements, such as Non-farm employment, noticeably increased. During the ZLB era (2007-2015), Crude Oil's responsiveness was comparatively subdued, indicating a unique interplay between monetary conditions and commodity volatility.
Gold reacts to macro news, especially the CPI and Personal Income, in the two sub-periods as well as in the full sample. Copper reacts comparatively less than gold in the full sample, but is more sensitive to news such as the CPI and Trade Balance during the ZLB sub-period.
%Gold consistently reacted to economic announcements across all periods, particularly to the "Consumer Price Index" and "Personal Income," reinforcing its role as a financial hedge. Copper exhibited variable sensitivity, being less reactive in the full sample but showing pronounced sensitivity to indicators like Consumer Price Index and Trade Balance during the ZLB period.
The impact of financialization, measured by $\phi_m$, is particularly significant for Gold and Palladium during the Covid period, while for Copper and Natural Gas the $\phi_m$ coefficient is more significant during the ZLB period. %This implies a trend toward increased financialization of these commodities in times of economic uncertainty. Interestingly, Copper and Natural Gas showed significant $\phi_m$ values during the ZLB era, emphasizing their heightened sensitivity to macroeconomic indicators during that period.
%%IL FAUT FAIRE ATTENTION AVEC LES INTERPRETATIONS. ON NE VEUT PAS SPECULER SANS AVOIR DE BONS ARGUMENTS. ACTUELLEMENT ON NE SAIT PAS VRAIMENT POURQUOI CERTAINES COMMO ET PAS D'AUTRES
The $R^2$ suggest a better model fit during the ZLB period, especially for Crude Oil and Natural Gas, but overall there are no meaningful differences between the sub-periods and the full sample.
%, indicating a more predictable impact of economic announcements during times of constrained monetary policy. The model was moderately effective during the Covid era and provided a broad baseline understanding in the full sample, encompassing the pre-Covid period.
\subsection{Discussion and implications}
%ICI IL Y A DES REFERENCES QUI DOIVENT ETRE MISES EN BIB TEXformat
%GOLDSTEIN C'EST PAS EVIDENCE C;EST UN MODEL

Our results provide a deeper understanding of price discovery in commodity markets and the role of different types of participants. Previous research, such as \citet{goldstein2022commodity}, has shown that  financial investors may have an adverse impact on commodity markets by introducing noise along with new information. However, our findings offer a different perspective by empirically assessing which types of financial participants seem to be responsible for introducing more noise to these markets.
%levels of risk aversion,
%Our analysis considers different types of traders, who may have different  economic objectives or regulatory restrictions.  


Unlike \citet{goldstein2022commodity}, we find that noise is not due to a concentration of financial participants, but rather to the rise of swap dealers in commodity markets. We find that the increased participation of financial investors, particularly money managers, improves price accuracy and reduces volatility. An increased participation by swap dealers, however, leads to less accurate prices and to greater volatility. Thus, if traders only consisted of money managers, an increase in their proportion beyond a certain threshold would continue to improve price accuracy and reduce volatility. However, if traders were composed only of swap dealers, prices would be less accurate and more volatile, regardless of whether the threshold was exceeded or not.
%Our findings indicate that the loss of accuracy and increase in volatility is not necessarily due to a high concentration of financial traders, but rather to a greater participation by one type of trader, specifically swap dealers. 
In addition, we find that volatility reacts less to announcement surprises when a commodity is more financialized, suggesting a benefit for traditional market participants. Our results are consistent with research including \citet*{brunetti2009speculation} and \citet*{cheng2015convective}, who have shown that fund managers, being more sensitive to market information, contribute to price discovery and liquidity in commodity markets.

%Overall, our contribution is to report new results that provide empirical evidence allowing us to better understand how information can be incorporated into the prices of commodity futures contracts. The increased participation of financial investors appears to improve price accuracy, as measured in terms of ``precision''.\footnote{Since precision is the inverse of variance, it can be argued that the reduction in variance by financial investors is equivalent to an increase in precision. More rigorously, the more the asset price has realized values that are scattered around the mean (high variance), the less accurate it is (low accuracy) and vice versa.} However, \citet{goldstein2022commodity} show in their model that, under certain circumstances, financial investors can be detrimental to commodity markets, as they also bring noise along with new information. 
 %
%Our second contribution is to determine which type of financial investors are responsible for the noise in this type of market. Contrary to \citet{goldstein2022commodity}, we find empirically that the noise is not simply due to the greater concentration of financial investors, but rather to the rise of swap dealers in this market.
 %
%
%We consider different financial trader types, who need not have the same level of risk aversion, the same economic objectives or the same regulatory restrictions. Based on our results, if financial traders consisted only of money managers, an increase in the proportion of financial traders past the critical threshold point \citep[see the model in ][]{goldstein2022commodity} would continue to improve price accuracy and thereby reduce volatility. On the other hand, if financial traders were composed only of swap dealers, prices would be less accurate and more volatile, whether or not the threshold was exceeded. Overall, we argue that the loss of accuracy (equivalently, the increase in volatility) is not necessarily due to a high concentration of financial traders, but rather to a greater participation by one type of trader, namely swap dealers.
%
%The precision of prices past the model's threshold point does not decrease to the level before the threshold point. This is consistent with our result showing that overall, financialization reduces volatility following a macroeconomic announcement. That is, price accuracy is still better past the threshold point compared to a situation where we have no financial traders and only commercial traders \citep[see also the model in ][]{kang2020tale}.
%
%As with \citet{brunetti2009speculation}, our first finding is based on using a financialization proxy that includes all financial investors. It is still possible for a specific class of trader to implement trading strategies that move prices and increase volatility. Knowing this, our results would imply that financialization as a whole reduces volatility when there is a macro announcement. We interpret this result as indicating that commodity markets are better informed--and macro news create less of a shock--when they are financialized, which should be beneficial for traditional commodity market participants. 
%
%Subsequently, we examine the impact of different types of traders by breaking down the data. We find that money managers seem to contribute to price discovery when there is a macro announcement, while helping to reduce volatility. This result is consistent with the fact that money managers are more informed investors, given their function in the markets. On the other hand, swap dealers also contribute to price discovery while causing an increase in volatility following macroeconomic announcements. 
%
%Our second result is consistent with \citet{cheng2012convective}, who show that fund managers are clearly more sensitive to market information and fill hedgers' liquidity needs by taking the opposite position. It is also consistent with  \citet{goldstein2014speculation} who show that financial speculators improve price informativeness, while hedgers decrease it. 