\section{Conclusion} \label{sec:conclusion}




This paper investigates the impact of financialization on the real economy through a new angle, namely high-frequency effects linked to macroeconomic announcements. We test empirically whether financialization has amplified the impact of macro announcement surprises on prices and volatility in commodity markets. Indeed, it is well known that equity and bond markets react to these surprises. If commodities behave more like financial assets due to financialization, we should expect commodity futures to display greater reactions to macro surprises. Rather than split our sample in two (pre- and post-financialization), we measure this variable by means of a time-varying and commodity-specific proxy. Our results suggest that financialization, by reducing volatility and improving price discovery, is beneficial to commodity markets. Indeed, a greater participation by financial actors does not appear to amplify the effects of macro announcement surprises on prices or volatility. On the contrary,  greater financialization in a given commodity has a dampening effect such that prices and volatility react less to macro surprises. This finding is consistent with information diffusion economic arguments.


Our results also find support in a literature suggesting that non-traditional investors, such as hedge funds, are beneficial to commodity markets by supplying liquidity, reducing volatility, and improving market efficiency. The results we present are robust to the use of a non-parametric variance estimator, different proxies for financialization, and to alternative empirical specifications (e.g., regression equation, high-frequency window, etc.). Lastly, by documenting a dampening effect on volatility shocks (thus reducing the real option value of delaying investments), this paper's findings suggest that financialization may also help with sustainability efforts in financing a green energy transition, alongside other instruments such as green bonds and portfolio screens for sustainable investments.