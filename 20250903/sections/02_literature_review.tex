\section{Literature Review}
\subsection{Financialization of commodities}
 \citet{basak2016model} develop a model which predicts that financialization will: (i) increase commodity futures prices, especially for futures that belong to the commodity index; (ii) increase volatility for both index and non-index futures; (iii) increase correlations between commodities and with equities; and (iv) only affect prices of storable commodities (e.g., wheat, crude oil).  Empirical evidence on their predictions is, however, inconclusive. Table \ref{tab:fin} presents a summary of research findings on the impact of financialization and speculation on commodity prices and volatility.\footnote{Note that we use a broader definition of financialization that what some authors use. In particular, we do not limit our analysis strictly to the impact of index traders.}   
%The literature has not reached a consensus concerning the impact of financialization on commodity futures prices and volatility. \citet{basak2016model} develop a theoretical model that yields four main findings suggesting a meaningful impact. (i) The prices of all commodity futures increase with financialization and this increase is more significant for futures belonging to the index than for non-indexed futures. (ii) The volatility of both index and non-index futures return increases with financialization. (iii) Correlations between commodity futures returns, as well as equity-commodity correlations, increase with financialization. (iv) Only prices of storable commodities are affected by financialization. Moreover, inventories and prices of storable commodities are higher in the presence of institutional investors than in the benchmark economy, and more so for commodities included in the index.

  \citet{tang2012index} report increases in the correlations of crude oil and non-energy commodity returns, which they claim are due to the rapid growth of index investments in commodity markets. They argue that when a commodity is included in a benchmark index, its price is no longer determined solely by the commodity's own supply and demand, but also by other commodities and assets in the index.   \citet{singleton2014investor} further shows that speculative activity by financial investors creates informational frictions, leading commodity prices to diverge from their fundamental value. This result implies that financialization would increase volatility. In a similar vein, \citet*{yang2005futures} show that when commodity futures trading volume increases, so does the volatility of commodity spot prices. 
%On  the one hand, some of the literature provides support for the predictions. 
%can lead to considerable increases in volatility, as a result of information friction and prices deviating from values consistent with the economics of the commodity.


Another line of research, however, finds no evidence that financialization is responsible for distorted prices or higher volatility. \citet*{brunetti2014commodity} use an equilibrium model and data on commodity trader positions to show that index traders provide insurance against price risk.     \citet*{brunetti2016speculators}  analyze the impact of particular types of speculators in commodity markets from 2005 to 2009. They find that hedge funds allow for a faster and more efficient price discovery, resulting in a lower volatility. Furthermore, they find that the positions of swap dealers are not correlated with contemporaneous returns and volatility in commodity markets.  
\citet*{stoll2010commodity} show that inflows and outflows from commodity index investments do not Granger-cause price or volatility changes. Using a no-arbitrage argument, \citet*{hamilton2014risk} show that the positions of commodity traders included in index funds cannot be used to achieve excess returns in futures markets. In addition, \citet*{kilian2014role} show that several speculative trades can occur in the oil market without seeing a significant change in inventory levels. This would seem to rule out speculation as being responsible for the boom and bust cycle in the oil market between 2003 and 2008. %\citet*{bryant2006causality} reject the hypothesis that speculation and uninformed traders affect volatility. They show that the two theories that would predict this hypothesis, hedging pressure and normal backwardation, are empirically rejected and have no explanatory power. Lastly, \citet*{bohl2013does} show that financialization implies no change in conditional variance for daily returns on raw material futures.



 %Some studies differentiate the impact of different trader types, given their economic motivations. \citet{brunetti2014commodity}  use an equilibrium model and data on trader positions in commodity futures markets to show that index traders provide insurance against price risk.   Closer to this paper, \citet{brunetti2016speculators}  analyze the impact of certain types of speculators in commodity markets from 2005 to 2009. They find that hedge funds allow for faster and more efficient price discovery, resulting in a lower volatility. Furthermore, they find that the positions of swap dealers are not correlated with contemporaneous returns and volatility in the commodity markets.  \textcolor{red}{Table 1} presents a summary of the literature on the influence of financialization and speculation on the volatility of commodity futures returns.
  %%However, a consensus has not been reached in the literature.%   As for financialization linked to speculation,   Moreover,  

\subsection{Macroeconomic announcements}


\subsubsection{Macro announcements and financial markets}

A large empirical literature documents the impact of macroeconomic announcements on stocks \citep*{,scholtus2014speed} and bonds \citep{fleming1997moves,fleming1999price}. This research is distinct from, but complementary to, a large body of work on informed trading and corporate announcements such as dividend changes \citep{zhang2018informed}.
%%% CE PARAGRAPHE POURRAIT ALLER DANS LA THESE MAIS PAS DANS LE PAPIER%%%%
%When a major economic release is issued, market participants examine the details to determine how the information in the report could affect the prices of stocks, bonds or commodity futures. When economic releases are better than expected (i.e., positive surprises),  stock prices should rise while bond prices should fall. Bond (or fixed income) yields rise as they are inversely related to prices. The reverse is true for a worse than expected economic release. Stock prices will fall on this news and bond prices will rise (yields will fall). On the commodity futures side, the impact of a macroeconomic announcement is more heterogeneous and depends on the particular commodity. %In the case of some commodities, prices rise when a macroeconomic announcement signals a weak economy, while prices fall for others. It is therefore important to consider what investors think in order to understand how they anticipate the impact of an announcement on the economy.
In a key study, \citet*{balduzzi2001economic} investigate the effects of scheduled macro announcements on prices, trading volume, and bid-ask spreads. They find that 17 public news releases have a significant impact on the prices of the three-month bill, the two-year and 10-year notes and a 30-year bond. There is a persistent and significant increase in volatility and trade volume following a macro announcement. The literature also looks at whether economic conditions can explain heterogeneous responses to announcements across business cycles. \citet{andersen2003micro}  find that the stock market reacts to news differently depending on the stage of the business cycle. \citet*{boyd2005stock} show that the impact of unemployment-related announcements varies according to the economic environment. 



%Boyd : While not directly related to commodities, it is informative in terms of expectations regarding macroeconomic announcements and what they represent for the economy. 

%A large empirical literature documents the impact of macroeconomic announcements on stocks \citep*{,scholtus2014speed} and bonds* \citep{fleming1997moves,fleming1999price}. In a seminal study, \citet*{balduzzi2001economic} investigate the effects of scheduled macroeconomic announcements on prices, trading volume, and bid-ask spreads. For each release, they measure the surprise in the announced quantity. Using intraday data from the interdealer government bond market, they find that 17 public news releases have a significant impact on the price of at least one of the following instruments: a three-month bill, a two-year note, a 10-year note, and a 30-year bond. Following a macro announcement, there is a persistent and significant increase in volatility and trade volume. The effects also vary by instrument maturity.
%
%Other research has examined whether economic conditions can explain some of the more heterogeneous responses to announcements across business cycles. \citet*{andersen2003micro} find that the stock market reacts to news differently depending on the stage of the business cycle.  \citet*{boyd2005stock} also shows that the impact of unemployment-related announcements varies according to the economic environment.  Exploiting the fact that high-frequency traders receive the Michigan Consumer Sentiment Index two seconds before its official announcement, \citet*{hu2017early} find evidence of highly concentrated trading and rapid price discovery occurring within 200 milliseconds. Outside this narrow window, the typical investor trades at fully adjusted prices. 
%The empirical analysis in  \citet*{boehm2020us} tracks the behavior of different financial assets following releases for different types of announcements in the US market. Using their results, we can explain which particular macroeconomic announcements send a signal of a weak economy when the sign of the surprise variable is positive. In our case, only announcements of the Consumer Price Index \citep*{bryan1993consumer,clark1997us} and Initial Jobless Claims \citep*{fleming1997moves,fleming1999price,getz1990barometer} signal a weaker economy when the surprise is positive. In other words, when the value of the macroeconomic announcement is better than expected, the surprise variable will have a negative sign. For the other announcements in our sample, we expect that a positive surprise will be interpreted by investors as signaling a strong economy. %In other words, a better than expected macroeconomic announcement value will be accompanied by a positive surprise variable.




%(see e.g. cite quelques papiers que j'ai enlevé)
\subsubsection{Macro announcements and commodity markets}

Much of the literature on announcements and commodity markets concerns production and inventory updates from the Organization of  Petroleum Exporting Countries (OPEC). Evidence on the impact of OPEC news on prices is mixed \citep{10.5089/9798400219788.001, 10.1016/j.enpol.2009.10.053}. While some research suggests a short-run impact on crude oil prices and volatility, OPEC's influence has weakened since the oil glut in the 1980s \citep{10.1111/j.1477-8947.1983.tb00276.x}.
%OPEC announcements are influenced by investor expectations and, in the short term, can have an impact on the oil price and its volatility  The ability of OPEC to influence commodity prices has been weakened in recent years due to the apparent oil glut in the early 1980s 
\citet*{horan2004implied} find that volatility drifts upward as OPEC meetings draw nearer and decreases in the first five days after the meetings start. \citet*{wirl2004impact} find little market reaction to fifty OPEC meetings from 1984 to 2001.   \citet*{kilian2011energy} find no significant responses in regressions of WTI crude oil and U.S. gasoline prices on 30 U.S. macro announcements from 1983 to 2008.   \citet*{gu2018drives} study the pre-announcement price drift in natural gas and show that inventory surprises can be predicted using the difference between the median analyst forecast (historically highly accurate) and the consensus forecast. 
As the lack of conclusive results in this area may be due to using daily data, we propose using high-frequency return data.
%The lack of conclusive results obtained by \citet*{wirl2004impact} and \citet*{kilian2011energy} may be due to the use of daily data for futures returns. We aim to overcome this problem by using high frequency returns. 
%to these meetings.assess the influence of OPEC on world oil markets by 
%They find no evidence of statistically significant responses for either oil or gasoline.
% to U.S. macroeconomic news at daily time horizons.
%vol decr by three percent and by five percent over a five-day window period


A smaller literature looks at how commodities react to macro announcements, also generally using daily data.  \citet*{frankel1985commodity} find that a positive shock in the money supply lowers gold prices. \citet*{christie2000macroeconomics} study how gold and silver futures prices react, over 15-minute intervals, to 23 U.S. macro announcements over 1992-1995 and find that volatility is higher on announcement days. \citet*{cai2001moves} analyze  five-minute gold price returns around macro announcements using GARCH models. They find that intraday price impacts of announcements are fewer and less significant for gold than for bonds or currencies.  Similarly, \citet*{hess2008commodity} find that CRB and Goldman Sachs commodity indexes are less sensitive to the impact of 17 U.S. macro announcements than are bonds or stocks.  \citet*{hollstein2020volatility} look at how different economic variables affect the term structure of commodity futures volatility. They show that speculation and jobs-related macro variables have the largest impact on volatility. Lastly, \citet*{ye2021macroeconomic}  find that volatility in commodity futures is more affected by macroeconomic forecasts than by current economic conditions. 

%So far, the literature shows a clear impact of macroeconomic announcements on the price of stocks and bonds. However, there does not seem to be a consensus on whether announcements impact the price of commodity futures. Our research provides some insight into this issue by expanding the set of announcements beyond those which are normally used, such as OPEC. The other major element that contributed to the significant results and is also at the center of our contribution is the addition of variables quantifying the level of financialization for each of the individual commodities studied.

So far, the literature shows a clear impact of macroeconomic announcements on stock and bond prices. However, there does not seem to be a clear answer to whether announcements affect commodity futures prices. Our research provides new  insights  by i) using high frequency data, ii) expanding the set of announcements beyond those which are normally analyzed (such as OPEC meetings) and iii) considering different variables to quantify the level of financialization for each of the commodities in the sample.